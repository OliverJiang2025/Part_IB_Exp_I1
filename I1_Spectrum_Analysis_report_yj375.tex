\documentclass[12pt]{article}[times]

\topmargin 0.0cm 
\oddsidemargin 0.0in 
\evensidemargin0.0in
\textheight 22cm 
\textwidth  17cm 
\headheight 0in 
\headsep 0in
\parindent0in


\usepackage{amsmath}
\usepackage{amsfonts}
\usepackage{amssymb}
\usepackage{graphicx}
\usepackage{tikz}
\usepackage{amssymb}
\usepackage{circuitikz}
\usepackage{setspace}
\doublespacing{}
\begin{document}

%\hspace{1cm}

\begin{center}
\Large{\bf CAMBRIDGE UNIVERSITY ENGINEERING DEPARTMENT}
\end{center}

\vskip 1cm

\begin{center}
\large{\bf Part IB Laboratory Report}
\end{center}




\vskip 1cm
\begin{center}
\fbox{\rule[0.0cm]{0cm}{1.0cm}
 \large{\bf I1 Spectrum Analysis}\rule[-0.75cm]{0cm}{1.0cm} }
\end{center}


\vskip 3cm

\begin{center}

Name: Yongqing Jiang \\
\vskip 0.2cm
 Lab Group No: 11 \\
\vskip 0.2cm
 College: Peterhouse \\
  \vskip 0.2cm

%
Date of Experiment: 3/3/2025
%
\end{center}

\vskip 3cm
\newpage

\section{Summary}
This experiment aims to introduce techniques on analyzing a
signal by spectrum analysis, during which signals are decomposed into 
their frequency components, revealing more information. This is done by a 
computer-based spectrum analyser during this experiment.
Amplitude Modulation (AM) is also analysed later in the experiment. At the same
time, demodulator design is also introduced.
\section{Readings and Results}
\subsection{Basic operation of picoscope software}
This section shows the basic operation methods of picoscope software. 
The waveform recorded by the picoscope might appear unstable. Triggering
can solve this problem as shown in Figure 1.
We can use the build-in spectrum analyser of picoscope software to 
obtain the spectrum of the recorded waveform as shown in Figure 2.
\noindent % Prevents indentation
\begin{figure}[h]
  \centering
  \begin{minipage}{0.45\textwidth}
    \includegraphics[width=\textwidth]{spectrum_analysis_data/5.1.png}
    \caption{Time domain of the example wave}
    \label{fig:image1}
  \end{minipage}
  \hfill % Space between the images
  \begin{minipage}{0.45\textwidth}
    \includegraphics[width=\textwidth]{spectrum_analysis_data/5.2.png}
    \caption{Frequency domain of the example wave}
    \label{fig:image2}
  \end{minipage}
\end{figure}
\newline
As we can see, frequency domain of the wave contains a peak which corresponds
to the value of the frequency of wave in time domain. 
Thus, we can see that it successfully decompose a signal in time domain
into its frequency components, and in this case, there is only one 
frequency component.

\subsection{Spectra of Simple Periodic Signals}
In this section, 2 iconic periordic signals is generated by the wave generator
and recorded by the picoscope. Their spectrum is generated by the software 
and compared to Fourier Series theories.
\subsubsection{Square wave}
For a perfect square wave with peak-to-peak amplitude 2 and period T 
has a Fourier Series representation of:
\begin{equation}
    x(t) = \sum^{\infty}_{n=1} b_n \sin{n\omega_0 t}
\end{equation}
Where $\omega_0 = \frac{2\pi}{T}$ is the fundamental frequency and 
the coefficient is found by 
\begin{equation}
    b_n = \frac{2}{T}\int^{\frac{T}{2}}_{-\frac{T}{2}}x(t)\sin{n\omega_0 t}dt =
    \begin{cases}
      \frac{4}{n\pi}, n = 1, 3, 5, \dots \\
      0, \text{ otherwise}
    \end{cases}
\end{equation}





\subsubsection{Triangular wave}
Fourier Series of a perfect triangular wave with peak-to-peak
amplitude 2 and period T is:
\begin{equation}
  x(t) = \sum^{\infty}_{n=1} b_n \sin{n\omega_0 t}
\end{equation}
with coefficient 
\begin{equation}
  b_n = \frac{2}{T}\int^{\frac{T}{2}}_{-\frac{T}{2}}x(t)\sin{n\omega_0 t}dt =
  \begin{cases}
    \frac{8}{\pi^2}\frac{(-1)^{(n-1)/2}}{n^2}, n = 1, 3, 5, \dots \\
    0, \text{ otherwise}
  \end{cases}
\end{equation}
\noindent % Prevents indentation
\begin{figure}[h]
  \centering
  \begin{minipage}{0.45\textwidth}
    \includegraphics[width=\textwidth]{spectrum_analysis_data/sq.png}
    \caption{Comparison of $b_n$ of square wave}
  \end{minipage}
  \hfill % Space between the images
  \begin{minipage}{0.45\textwidth}
    \includegraphics[width=\textwidth]{spectrum_analysis_data/tri.png}
    \caption{Comparison of $b_n$ of triangular wave}
  \end{minipage}
\end{figure}
\newpage
As shown in Figure 3 and 4, the first several harmonics have measured value
highly agree to theoretical values. However, the plot seems to 
deviate a lot from n = 6 in square wave plot and n = 4 in triangular
wave. This is because a decibel scale is used and when numerical
values gets small, the error is exponentially enlarged. When n goes up,
as given by the expression of coefficients, $b_n$ will become smaller value
and is more prone to be affected by noise present. Thus, this deviation
is within expectation.
\newline
To provide more insight on the hidden relations between square
wave and triangular wave, it can be stated from the fact that 
two square waves can be convolved to be a triangular wave.

\subsection{Amplitude Modulation}
Assume the information signal is:
\begin{equation}
  x(t) = E\cos(\omega t)
\end{equation}
This signal can be modulated by a carrier signal:
\begin{equation}
  x_c(t) = E_c \cos(\omega_c t)
\end{equation}
The resultant modulated signal is composed of carrier signal and
a offsetted information signal.
\begin{equation}
  x_m(t) = x_c(t) + x(t)\cos(\omega_c t) \\
  = E_c(1+m\cos{\omega t})\cos{\omega_c t}
\end{equation}
In which m is the modulation ratio defined by $\frac{E}{E_c}$. This
expression can be expanded to show its frequency components:
\begin{equation}
  x_m(t) = E_c \cos{\omega_c t} + \frac{mE_c}{2}\cos{(\omega_c + \omega)t} + \frac{mE_c}{2}\cos{(\omega_c - \omega)t}
\end{equation}

Amplitgude modulation is widely used in signal transmission, especially
for many band-limited signals over common mediums. Band-limited
signals have such a small bandwidth that they are hard to be
transmitted over a long distance and have much higher chance to 
interfere with other signals with close range of frequency. 
\newline
However, AM provides a tool to change the frequency content of 
information signal to a higher value determined by carrier wave.
Besides, demodulation of AM is simple to be conducted.

\begin{figure}[h]
  \centering
  \includegraphics[width=0.8\textwidth]{spectrum_analysis_data/7.1.png}
  \caption{Time and frequency domain of AM signal}
\end{figure}
To adjust the waveform to be similar to the shape as shown in
Figure 4 in the handout, the modulator settings are: 
\begin{enumerate}
  \item carrier amplitude = 4.3V
  \item AM depth = 50$\%$
\end{enumerate}
To find the modulation index, it can be noted that 
the ratio of maximum amplitude of the envelope to the 
minimum is $\frac{1+m}{1-m}$. In this case, the ratio is approximately 3.
Thus, $m = \frac{1}{2}$.
\newline
For the case $m=1$, $E_{min} = min_t\{E_c(1+m\cos{\omega t})\} = 0$.
So we can see how value of m affects the shape of the envelope of 
the modulated wave. We can adjust parameters of the modulator so that 
this wave can be simulated. 
In order to achieve the value of m to be 1, we can adjust the
modulator parameter to:
\begin{enumerate}
  \item carrier level = 3V
  \item modulation depth = 100$\%$
\end{enumerate}
The resulting wavefrom is shown in Figure 6.

However, this simple AM can not represent the original
information signal well since there is a carrier 
frequency component in the spectrum. To eliminate this carrier
frequency component, Double-SideBand Suppressed Carrier (DSB-SC)
wave can be used. This wave is shown in Figure 7.
\noindent % Prevents indentation
\begin{figure}[h]
  \centering
  \begin{minipage}{0.45\textwidth}
    \includegraphics[width=\textwidth]{spectrum_analysis_data/7.2.png}
    \caption{AM wave with m=1}
  \end{minipage}
  \hfill % Space between the images
  \begin{minipage}{0.45\textwidth}
    \includegraphics[width=\textwidth]{spectrum_analysis_data/7.3.png}
    \caption{DSB-SC wave}
  \end{minipage}
\end{figure}
As shown in Figure 7, the carrier frequency component is eliminated.
In this case, $\frac{1+m}{1-m}=\frac{3}{0.15}$, m is found to be $\frac{19}{21}=0.905$.



\newpage
\subsection{Amplitude Demodulation}
A simple demodulator circuit is shown in Figure 8, which is 
composed of a diode, a RC charging and discharging circuit.
\noindent % Prevents indentation
\begin{figure}[h]
  \centering
  \begin{minipage}{0.45\textwidth}
    \includegraphics[width=\textwidth]{spectrum_analysis_data/demodulator.png}
    \caption{A demodulator circuit}
  \end{minipage}
  \hfill % Space between the images
  \begin{minipage}{0.45\textwidth}
    \includegraphics[width=\textwidth]{spectrum_analysis_data/demod_wave.png}
    \caption{Resulting demodulator waveform}
  \end{minipage}
\end{figure}
\newline
During demodulation process, the RC keep switching between charging
and discharging state. The output voltage is given by:
\begin{equation}
  v(t) = \begin{cases}
  v(\tau)e^{-\frac{t-\tau}{RC}}, \text{ if } x_m(t) < v(\tau)e^{-\frac{t-\tau}{RC}} \\
  x_m(t), \text{ otherwise}
  \end{cases}
\end{equation}
The charging and discharging process rates are dependent on the 
value of time constant of the RC circuit, the product of resistance
and capacitance, RC. When RC has a large value, the process will 
be really slow. Vice versa, the process is really fast. In order 
to achieve effective reconstruction, the value of RC should be 
as small as possible so that charging and discharging processes
are fast and the resultant curve is close to $x_m(t)$.

\begin{figure}[h]
  \centering
  \begin{minipage}{0.45\textwidth}
    \includegraphics[width=\textwidth]{spectrum_analysis_data/8.1A.png}
    \caption{Waveform at TP1}
  \end{minipage}
  \hfill % Space between the images
  \begin{minipage}{0.45\textwidth}
    \includegraphics[width=\textwidth]{spectrum_analysis_data/8.2B.png}
    \caption{Waveform at TP2}
  \end{minipage}
\end{figure}

The recorded waveforms at Test Point 1 and Test Point 2 are 
shown in Figure 10 and 11.



\section{Conclusion}

In the experiment, theories on spectrum analysis and amplitude 
modulation are derived and 
tested over a picoscope, a wave generator, and a modulator.
From the experiment results, the observations agree well to the
theories and provide more insight to future applications of 
spectrum analysis.





\end{document}